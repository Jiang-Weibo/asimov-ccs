\documentclass[11pt]{article}

\usepackage{amsmath}
\usepackage{epcc}
\usepackage{color}
\usepackage{marginnote}
\usepackage[pdftex]{graphicx}
\usepackage{subcaption}
\usepackage{newfloat}
\usepackage{bm}
\usepackage{booktabs}
\usepackage{hyperref}
\usepackage[svgnames]{xcolor}
  \definecolor{diffstart}{named}{Grey}
  \definecolor{diffincl}{named}{Green}
  \definecolor{diffrem}{named}{Red}
\usepackage{listings}
  \lstdefinelanguage{diff}{
    basicstyle=\ttfamily\small,
    morecomment=[f][\color{diffstart}]{@@},
    morecomment=[f][\color{diffincl}]{+\ },
    morecomment=[f][\color{diffrem}]{-\ },
  }

\hypersetup{colorlinks}
\usepackage{float}

\hyphenpenalty=500

% This is for ``figures'' which are to have a caption
% labelled ``Benchmark''.
\DeclareFloatingEnvironment[
  name=Benchmark,
  placement=tbhp,
  within=section,
]{benchmark}


\definecolor{terminalcolour}{gray}{0.96}

% Code fragments are gray...
\lstdefinestyle{codefragment}{
  basicstyle=\small\ttfamily,
  backgroundcolor=\color{terminalcolour},
  xleftmargin=0pt
}

% Benchmark tables
\lstdefinestyle{terminalverbatim}{
  basicstyle=\small\ttfamily,
  xleftmargin=0pt
}

\lstset{showstringspaces=false}

\renewcommand{\refname}{Notes and References}


\begin{document}
\lstset{style=codefragment}

\title{ASiMoV CCS}

% \date{July 2021}
\author{EPCC}

\makeEPCCtitle

\centerline{\sc Testing guide}

\tableofcontents
\pagebreak

\bigskip

%\hrule

\bigskip

% CONTENT HERE

\section{Introduction}

\subsection{Property-based testing}
\label{subsec:prop-based-test}

A challenge of testing numerical methods is knowing \textit{what is the right answer?} as typically
a numerical method is used when a problem cannot be solved analytically.
Even when a problem can be solved in a simple case the numerical solution is an approximation and
the question becomes \textit{is this answer good enough?}

Rather than test specific examples such as
\begin{equation}
  f_i=1, f_{i+1}=2, \Delta{}x=1 \Rightarrow \left. \frac{\delta f}{\delta x} \right|_{i+1/2} = 1
\end{equation}
property-based testing ensures that an implementation of a function (such as approximating a
derivative) maintains certain properties, without necessarily needing to know the correct answer ---
see table~\ref{tab:cont-vs-disc-derivs} for an example of properties a discrete derivative
approximation should maintain.
An additional benefit of this approach is that properties can be tested over a wide range of
randomly generated inputs, achieving much broader testing than would be possible with specific
examples such as above.

\section{Discretisation schemes}

For a given discretisation scheme, the properties displayed in table~\ref{tab:cont-vs-disc-derivs}
are expected to hold

\begin{table}[h]
  \centering
  \caption{Comparison of properties of continuous and discrete
    derivatives}\label{tab:cont-vs-disc-derivs}
  \begin{tabular}[h]{c|cc}
    Property & Continuous & Discrete \\
    \hline
    Negation & $\dfrac{\partial \left(-f\right)}{\partial x} = -\dfrac{\partial f}{\partial x}$ & Y
    \\
    Scaling & $\dfrac{\partial \alpha f}{\partial x} = \alpha \dfrac{\partial f}{\partial x},\
              \kappa=const$ & Y \\
    Summation & $\dfrac{\partial \left( \alpha f + \beta g \right)}{\partial x} = \alpha
                \dfrac{\partial f}{\partial x} + \beta \dfrac{\partial g}{\partial x}$ & Y \\
    Exactness & --- & $\dfrac{\delta f^n\left(x\right)}{\delta x} = \dfrac{\partial
                    f^n\left(x\right)}{\partial x},\ n < \mathcal{P}$
  \end{tabular}
\end{table}

note that the ``Exactness'' property means that for low-order schemes the product rule only applies
in the special case that one of the variables is a constant, in which case it reduces to the
``Scaling'' property.
Analogous properties also hold for interpolations (the two being related).

\section{Mesh}

\subsection{Mesh partitioning and ordering}

These are both in some sense sorting methods, so should maintain similar properties to a sorting
scheme:
\begin{itemize}
\item a sorted list contains the same elements as an unsorted list
\item sorting a sorted list should do nothing\footnote{This may not work for mesh
    partitioning/ordering as often heuristics are involved.}
\end{itemize}

\end{document}
%%% Local Variables:
%%% mode: latex
%%% TeX-master: t
%%% End:

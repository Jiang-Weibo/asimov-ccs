\documentclass[11pt]{article}

\usepackage{epcc}
\usepackage{color}
\usepackage[pdftex]{graphicx}
\usepackage{subcaption}
\usepackage{newfloat}
\usepackage{bm}
\usepackage{booktabs}
\usepackage{hyperref}
\usepackage[svgnames]{xcolor}
  \definecolor{diffstart}{named}{Grey}
  \definecolor{diffincl}{named}{Green}
  \definecolor{diffrem}{named}{Red}
\usepackage{listings}
  \lstdefinelanguage{diff}{
    basicstyle=\ttfamily\small,
    morecomment=[f][\color{diffstart}]{@@},
    morecomment=[f][\color{diffincl}]{+\ },
    morecomment=[f][\color{diffrem}]{-\ },
  }

\hypersetup{colorlinks}
\usepackage{float}

\hyphenpenalty=500

% This is for ``figures'' which are to have a caption
% labelled ``Benchmark''.
\DeclareFloatingEnvironment[
  name=Benchmark,
  placement=tbhp,
  within=section,
]{benchmark}


\definecolor{terminalcolour}{gray}{0.96}

% Code fragments are gray...
\lstdefinestyle{codefragment}{
  basicstyle=\small\ttfamily,
  backgroundcolor=\color{terminalcolour},
  xleftmargin=0pt
}

% Benchmark tables
\lstdefinestyle{terminalverbatim}{
  basicstyle=\small\ttfamily,
  xleftmargin=0pt
}

\lstset{showstringspaces=false}

\renewcommand{\refname}{Notes and References}


\begin{document}
\lstset{style=codefragment}


\title{ASiMoV CCS}

\date{July 2021}
\author{EPCC}

\makeEPCCtitle

\centerline{\sc Programming style guide}

\tableofcontents
\pagebreak

\bigskip

%\hrule

\bigskip

% CONTENT HERE

\section{Introduction}
TODO - what is this document

\section{Fortran standard}
TODO - code must follow Fortran 2018 standard

\section{Naming conventions and capitalisation}
All Fortran keywords must be written in lower case.

\section{Layout and formatting}
The following is a list of recommended practices for layout and formatting for ASiMoV CCS.

\begin{itemize}
\item Templates are provided for modules, submodules, functions, subroutines, and programs. The templates should always be as a starting point for new code.
\item Indent blocks by 2 spaces. Where possible, comments should be indented with the code within a block.
\item Use space and blank lines where appropriate to format your code to improve readability.
\item Where possible, avoid using continuation lines in a statement.
\item Avoid putting multiple statements on the same line.
\item Each program unit should follow a defined structure. The intended behaviour of the unit should be clearly described in the header.
\item Comments should start with a single ! at beginning of the line. A blank line should be left before (but not after) the comment line. The exception is for one line comments which can be indented within the code or placed after the statement.
\item Each module should be in a separate file.
\end{itemize}

\section{Style}

\subsection{Array notation}
In order to make it clear that variables used in an assignment are arrays, always use array notation.

Common practice, bad for readability:
\begin{lstlisting}[
	language=fortran,
    basicstyle=\fontsize{7}{10}\ttfamily
    \label{lst:array_bad}
]
integer :: a(10, 20), b(10, 20)
a = b ! Bad
\end{lstlisting}

Better practice, improved readability:
\begin{lstlisting}[
	language=fortran,
    basicstyle=\fontsize{7}{10}\ttfamily
    \label{lst:array_good}
]
integer :: a(10, 20), b(10, 20)
a(:) = b(:) ! Good
\end{lstlisting}

\subsection{Parentheses}
Improve readability of mathematical formulas by explicity using parentheses.

Common practice, bad for readability:
\begin{lstlisting}[
	language=fortran,
    basicstyle=\fontsize{7}{10}\ttfamily
    \label{lst:parens_bad}
]
integer :: a, b, c, d
a = b * c + d ! Bad
\end{lstlisting}

Better practice, improved readability:
\begin{lstlisting}[
	language=fortran,
    basicstyle=\fontsize{7}{10}\ttfamily
    \label{lst:parens_good}
]
integer :: a, b, c, d
a = (b * c) + d ! Good
\end{lstlisting}

\subsection{Logical expressions}
True/false conditional statements should use logical rather than relational expressions (i.e. with variables of 
type \texttt{logical} rather than integers). 

Common practice, bad for readability:
\begin{lstlisting}[
	language=fortran,
    basicstyle=\fontsize{7}{10}\ttfamily
    \label{lst:parens_bad}
]
integer :: n
n = 1
if(n) then ! Bad
  write(*,*) "condition is true"
end if
\end{lstlisting}

Better practice for readability:
\begin{lstlisting}[
	language=fortran,
    basicstyle=\fontsize{7}{10}\ttfamily
    \label{lst:parens_bad}
]
integer :: n
n = 1
if(n == 1) then ! Better
  write(*,*) "condition is true"
end if
\end{lstlisting}

Good practice - unambiguous style for improved readability:

\begin{lstlisting}[
	language=fortran,
    basicstyle=\fontsize{7}{10}\ttfamily
    \label{lst:parens_good}
]
logical :: n
n = .true.
if(n .eqv. .true.) then ! Good
  write(*,*) "condition is true"
end if
\end{lstlisting}



Comparisons involving \texttt{logical} variables should use the 
following logical expression operators:
\begin{lstlisting}[
	language=fortran,
    basicstyle=\fontsize{7}{10}\ttfamily
    \label{lst:logic_op}
]
.and.  ! logical conjunction: true if both A and B are true. 
.eqv.  ! logical equivalence: true if both A and B are true, or both are false. 
.neqv. ! logical inequivalence (exclusive OR): true if either A or B is true, but false if both are true.  
.not.  ! logical negation: true if A is false and false if A is true. 
.or.   ! logical disjunction (inclusive OR): true if either A, B, or both, are true.  
.xor.  ! same as .neqv. 
\end{lstlisting}

\subsection{Relational expressions}
Relational expressions (i.e. not involving \texttt{logical} variables) should use the following operators:
\begin{lstlisting}[
	language=fortran,
    basicstyle=\fontsize{7}{10}\ttfamily
    \label{lst:logic_op}
]
<   ! less than
<=  ! less than or equal to
>   ! greater than
>=  ! greater than or equal to
==  ! equal to
/=  ! not equal to
\end{lstlisting}

\subsection{Spaces between keywords}
The following keywords should be written with spaces, rather than as a single word:
\begin{lstlisting}[
	language=fortran,
    basicstyle=\fontsize{7}{10}\ttfamily
    \label{lst:parens_good}
]
else if
end if
end do
end forall
end function
end subroutine
end interface
end select
end type
end where
end module
end program
select case
\end{lstlisting}

\end{document}

